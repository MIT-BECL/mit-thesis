%% This is an example first chapter.  You should put chapter/appendix that you
%% write into a separate file, and add a line \include{yourfilename} to
%% main.tex, where `yourfilename.tex' is the name of the chapter/appendix file.
%% You can process specific files by typing their names in at the 
%% \files=
%% prompt when you run the file main.tex through LaTeX.
\chapter{Introduction}

Microbes, the oldest and most diverse type of life on Earth, are essential to
human life and industry. Microbes catalyze key biogeochemical cycles, clean up
toxic pollution, and can prevent or cure disease. Microbial ecology---the study of
microbes' relationships with one another and their environment---has
benefited tremendously from nucleic acid sequencing technology. RNA was first
sequenced in the 1960s~\cite{sanger_two_1965}. Microbiologists, notably Carl
Woese, used sequences of 16S rRNA, present in all bacteria and archaea, to
develop a more accurate ``tree'' of microbial
life~\cite{woese_phylogenetic_1977,pace_phylogeny_2012}. DNA was first
sequenced in the 1970s, and 16S rRNA gene sequences were directly amplified
from the environment and sequenced for the first time in
1990~\cite{giovannoni_genetic_1990,case_use_2007}. High-throughput sequencing
and sample multiplexing~\cite{hamady_error_2008} have enabled microbial
ecologists to collect large datasets at a decreasing cost of time and money.

Even as tools to process nucleic acid sequence data for microbial ecology
mature, there remain challenges to interpreting sequence data and integrating
them with other types of microbial ecology data. In this thesis, I present
the results of three projects that aim, in part, to help interpret microbial
sequence data. In Chapter 2, I discuss the Treatment Effect eXplorer for
Microbial Ecology Experiments (\texttt{texmex}), an analytical tool designed to
extract information about the dynamics of microbial taxa from microbial
ecology experiments with few timepoints and few or no replicates. In Chapter 3,
I discuss a novel biogeochemical model of a lake ecosystem, a novel
network-analysis method for organizing microbial ecology sequence data,
and intepretative frameworks for linking those models and methods with
survey data and a single-cell genetic assay. In Chapter 4, I discuss an
``omics-free'' model designed to guide the design and execution of clinical
trials that use fecal microbiota transplants. In the rest of this chapter,
I introduce and contextualize these contributions; in the final chapter,
I discuss the limitations and potential extensions of theses studies.

\section{Interpreting microbial ecology sequence data}
A major challenge in analyzing and interpreting microbial ecology experiments
that use sequencing is to determine which microbial taxa are meaningfully
different in abundance. Most approaches to this question are statistical and
focus, one-by-one, on the abundances of each taxon, analogous to methods that
identify differentially expressed genes in transcriptomic data (e.g.,
DESeq~\cite{anders_high_2010}). The $t$-test and simple ordinations have been
mainstays of this type of analysis for microbial ecology sequence
data~\cite{segata_metagenomic_2011}.

This approach presents difficulties when there are few or no biological
replicates, compositional effects affect measurements of composition, or when
it is important to integrate pre-test samples. I therefore developed
\texttt{texmex}, which considers the distribution of abundances of microbial taxa
\emph{within} a sample to make a more coherent comparison of abundances across
samples. I noted that microbial abudances within a sample follow the Poisson
lognormal distribution, which had been previously studied in macroecology, and
used that distribution to ``normalize'' the abundances of microbial taxa
within samples before comparing abundances across samples. As described in
Chapter 2, I applied this technique to short timeseries experiments measuring
the response of ocean microbial communities to amendment with crude oil,
verifying the presence of known oil degraders and suggesting that other
organisms, implicated in other crude oil microcosm experiments, are also
involved in crude oil degradation.

Microbial ecology surveys that use taxonomic marker gene sequence data
confront a different problem: the relationship between phylogeny and
function. Taxonomic marker gene sequencing (e.g., 16S rRNA amplicon sequencing)
provides information about microbial community structure but provides
only indirect information about microbial community function~\cite{langille_predictive_2013}. Metagenomic
sequencing provides more direct information about community function,
but it is expensive and mostly discards information about the relationships
between phylogeny and function.

In Chapter 3, I describe a set of tools used to identify potential consortia
of cooperating organisms in a natural ecosystem. This study includes a
taxonomic survey of the environment, an ecosystem-level model of microbial
metabolisms, and a single-cell genetic assay that links phylogeny and
function. For that study, I developed a network-analysis method (operational
ecological units) for the survey data and an interpretative framework
(inferred biomass) for the model result. These tools provided a conceptual
link between all three types of data, which were together essential for
the putative \textit{in situ}, perturbation-free identification of
microbial consortia.

\section{Clinical trial design}
Fecal microbiota transplantation (FMT) is a highly effective intervention for
patients suffering from recurrent \textit{Clostridium difficile}, a
common nosocomial infection~\cite{kassam_fecal_2013}. As the gut microbiome is
assigned an increasingly important role in the development and maintenance of host
gut health, immunity, and even psychology, the success of FMT for \textit{C.
difficile} has generated interest in using FMT to treat other conditions to
which an ``unhealthy'' gut microbiota may
contribute~\cite{smits_therapeutic_2013}. Although the exact mechanism by which FMT
treats \textit{C. difficile} is not well understood, a few clinical
trials have already experimented with using FMT as a treatment for other
conditions, and many more trials for a variety of diseases will probably begin
in the coming years.

These early trials using FMT to treat conditions beyond \textit{C. difficile}
have had some confusing and disappointing results. Of two recent
trials using FMT to treat inflammatory bowel disease, one
failed~\cite{rossen_findings_2015} and the other produced an unexpected
result: five of six stool donors in the trial appeared to produce results no
better than treating the patients with placebo, while the sixth donor produced
stool that appeared to have substantially greater efficacy~\cite{moayyedi_fecal_2015}.

If this sort of difference in efficacy between donors' stool is real, then
what are the implications for clinical trial design? 
To answer this question, I designed a model of differences in stool
efficacy and used the model to predict the statistical power of typical trial
designs. These results, laid out in Chapter 4, show that, if the results
from this recent inflammatory bowel disease trial are representative, then
FMT's efficacy is only likely to be discovered using larger patient cohorts
and response-adaptive allocation of donors' stool. I used this model
to aid the design of a two-stage phase I clinical trial using FMT to treat
a gastrointestinal condition, and I expect that these power calculations and
adaptive allocation strategies will improve clinical trial design, improving
the probability that patients will have access to a new therapy.

\begin{singlespace}
\bibliography{main}
\bibliographystyle{unsrt}
\end{singlespace}
